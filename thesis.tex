\documentclass[cs6size,a4paper]{ctexart}   
%==================== 数学符号公式 ============
\usepackage{amsmath}                 % AMS LaTeX宏包
\usepackage[style=1]{mdframed}
\usepackage{amsthm}
\usepackage{pdfpages}
\usepackage{amsfonts}
\usepackage{mathrsfs}                % 英文花体字 体
\usepackage{bm}                      % 数学公式中的黑斜体
\usepackage{manfnt}           % 一些图标,如 \dbend
\usepackage{lettrine}                % 首字下沉,命令\lettrine
\def\attention{\lettrine[lines=2,lraise=0,nindent=0em]{\large\textdbend\hspace{1mm}}{}}
\usepackage{longtable}
\usepackage[toc,page]{appendix}
\usepackage{geometry}                % 页边距调整
\geometry{top=3.0cm,bottom=2.7cm,left=2.5cm,right=2.5cm}
%====================公式按章编号==========================
\numberwithin{equation}{section}
\numberwithin{table}{section}
\numberwithin{figure}{section}
%================= 基本格式预置 ===========================
\usepackage{fancyhdr}
\pagestyle{fancy}
\fancyhf{}  
\fancyhead[C]{\zihao{5}  \kaishu 这是一份实验报告}
\fancyfoot[C]{~\zihao{5} \thepage~}
\renewcommand{\headrulewidth}{0.65pt} 
\CTEXsetup[format={\centering\bfseries\zihao{-2}},name={第, 章}]{section}
\CTEXsetup[nameformat={\bfseries\zihao{3}}]{subsection}
\CTEXsetup[nameformat={\bfseries\zihao{4}}]{subsubsection}

%================== 表格 =========================
\usepackage{makecell,rotating,multirow,diagbox}
%==================  图片 =========================
\newcommand{\addfig}[4][0.5]
{
  \begin{figure}[thbp!]
  \centering\includegraphics[width=#1\linewidth]{figure/#2}
  \caption{#4}
  \label{#3}
  \end{figure}
}
%================== 图形支持宏包 =========================
\usepackage{subfigure}
\usepackage{graphicx}                % 嵌入png图像
\usepackage{color,xcolor}            % 支持彩色文本、底色、文本框等
\usepackage{hyperref}                % 交叉引用
%\usepackage{caption}
\usepackage[font=small,labelfont=bf,labelsep=space]{caption}
\captionsetup{figurewithin=section}
%==================== 源码和流程图 =====================
\newcommand{\addcode}[2][None]
{
	\lstinputlisting[language=#1]{code/#2}
}

\usepackage{listings}                % 粘贴源代码
\usepackage{xcolor}
\usepackage{color}
\definecolor{dkgreen}{rgb}{0,0.6,0}
\definecolor{gray}{rgb}{0.5,0.5,0.5}
\definecolor{mauve}{rgb}{0.58,0,0.82}
\usepackage{xcolor}

\definecolor{dkgreen}{rgb}{0,0.6,0}
\definecolor{gray}{rgb}{0.5,0.5,0.5}
\definecolor{comment}{rgb}{0.56,0.64,0.68}
\lstset{
  frame=tb,
  aboveskip=3mm,
  belowskip=3mm,
  xleftmargin=2em,
  xrightmargin=2em,
  showstringspaces=false,
  keepspaces=true,
 columns=flexible,
  framerule=1pt,
  rulecolor=\color{gray!35},
  backgroundcolor=\color{gray!5},
  basicstyle={\small\ttfamily},
  numbers=none,
  numberstyle=\tiny\color{gray},
  keywordstyle=\color{blue},
  commentstyle=\color{comment},
  stringstyle=\color{dkgreen},
  breaklines=true,
  breakatwhitespace=true,
  tabsize=2,
}
% \lstset{
% breaklines=true,
%  %行号
%    numbers=left,
%     numberstyle= \tiny, 
%    basicstyle=\tiny,
%    %背景框
%    framexleftmargin=8mm,
%    frame=none,
%     %背景色
%    %backgroundcolor=\color[rgb]{1,1,0.76},
%     backgroundcolor=\color[RGB]{245,245,244},
%     %样式
%   keywordstyle=\bf\color{blue},
%   identifierstyle=\bf,
%    numberstyle=\color[RGB]{0,192,192},
%    commentstyle=\it\color[RGB]{0,96,96},
%   stringstyle=\rmfamily\slshape\color[RGB]{128,0,0},
%   %显示空格
%    showstringspaces=false
% }


%--------------------
\hypersetup{hidelinks}
\usepackage{booktabs}  
\usepackage{shorttoc}
\usepackage{tabu,tikz}
\usepackage{float}

\usepackage{multirow}



\tabcolsep=1ex
\tabulinesep=\tabcolsep
\newlength\tikzboxwidth
\newlength\tikzboxheight
\newcommand\tikzbox[1]{%
        \settowidth\tikzboxwidth{#1}%
        \settoheight\tikzboxheight{#1}%
        \begin{tikzpicture}
        \path[use as bounding box]
                (-0.5\tikzboxwidth,-0.5\tikzboxheight)rectangle
                (0.5\tikzboxwidth,0.5\tikzboxheight);
        \node[inner sep=\tabcolsep+0.5\arrayrulewidth,line width=0.5mm,draw=black]
                at(0,0){#1};
        \end{tikzpicture}%
        }

\makeatletter
\def\hlinew#1{%
  \noalign{\ifnum0=`}\fi\hrule \@height #1 \futurelet
   \reserved@a\@xhline}
   
\newcommand{\tabincell}[2]{\begin{tabular}{@{}#1@{}}#2\end{tabular}}%

\usepackage{subfigure}

\usepackage{CJK}
\usepackage{ifthen}


\usepackage{graphicx} 
\newcommand{\HRule}{\rule{\linewidth}{0.5mm}}

\newtheorem{Theorem}{定理}
\newtheorem{Lemma}{引理} 
%%使得公式随章节自动编号
\makeatletter
\@addtoreset{equation}{section}
\makeatother
\renewcommand{\theequation}{\arabic{section}.\arabic{equation}}

%-------------------------
	
\usepackage{pythonhighlight}
\usepackage{tikz}                    
\usepackage{tikz-3dplot}
\usetikzlibrary{shapes,arrows,positioning}
%===================   正文开始    ===================
\begin{document}
\bibliographystyle{gbt7714-2005}     %论文引用格式
%===================  定理类环境定义 ===================
\newtheorem{example}{例}              % 整体编号
\newtheorem{algorithm}{算法}
\newtheorem{theorem}{定理}            % 按 section 编号
\newtheorem{definition}{定义}
\newtheorem{axiom}{公理}
\newtheorem{property}{性质}
\newtheorem{proposition}{命题}
\newtheorem{lemma}{引理}
\newtheorem{corollary}{推论}
\newtheorem{remark}{注解}
\newtheorem{condition}{条件}
\newtheorem{conclusion}{结论}
\newtheorem{assumption}{假设}
%==================重定义 ===================
\renewcommand{\contentsname}{目录}     
\renewcommand{\abstractname}{摘要} 
\renewcommand{\refname}{参考文献}     
\renewcommand{\indexname}{索引}
\renewcommand{\figurename}{图}
\renewcommand{\tablename}{表}
\renewcommand{\appendixname}{附录}
\renewcommand{\proofname}{证明}
\renewcommand{\algorithm}{算法} 
%============== 封皮和前言 =================
\includepdf[page=1]{cover/cover}
\pagestyle{plain}
\pagenumbering{roman}
\section*{\zihao{3} \centering 摘要}

\vskip0.5cm
该实验包含四个自实验,包括基础的交换机组网,验证广播风暴, 验证vlan, 以及用两个路由器通过\emph{sub ip}实现跨网段

\textbf{关键词:}  交换机, 组网, 广播风暴, vlan, sub ip
\addcontentsline{toc}{section}{摘要}

\clearpage
\section*{\zihao{2} \centering \textbf{Abstract} }
The experiment consists of four self-experiments, including basic switch networking, verifying broadcast storms, verifying vlan, and using two routers to implement cross-network segments through sub ip

   %用了Times New Roman字体来美化观感
\textbf{Key Words:}switch, building network, broadcast storm, vlan, sub ip
\addcontentsline{toc}{section}{Abstract}





%\pagestyle{empty}
\tableofcontents 
%\thispagestyle{empty}
%============== 论文正文   =================
\pagestyle{fancy}
\pagenumbering{arabic}

\section{示例}

\subsection{插入图片}
如图\ref{fig:logo}所示,这是一个滑稽\\
%插入图片[图片宽度/页宽]{path}{label}{caption}
\addfig[0.2]{logo.jpg}{fig:logo}{滑稽的基类}

\subsection{插入代码}
\addcode[python]{code.py}

\subsection{插入数学公式}
%https://zh.numberempire.com/latexequationeditor.php, 在这个网站上生成公式代码
$$ \frac{\partial f}{\partial x} = 2\,\sqrt{a}\,x $$

\subsection{插入表格}
如\ref{table}所示, 这是一个表格
\begin{table}[H]
 \centering 
\begin{tabular}{|c|*{4}{c}|}
\hline
\diagbox{序号1}{序号2} & A & B & C& D \\
\hline
数字 & 1 & 2 & 3 & 4 \\
\hline
数字 & 2 & 4 & 6 & 8 \\
\hline
\end{tabular}
\caption{表格}\label{table}  
\end{table}


\subsection{关于缩进}
\indent \textbf{xxxx}\\
\noindent xxxx

\subsection{list}
\begin{enumerate}
	\item  啊啊啊啊啊
	\item  嗯嗯嗯嗯恩???
\end{enumerate}
\begin{description}
  \item[牛逼的人] linus, stallman 
  \item[牛逼的工具] git, gnuplot
\end{description}
\begin{itemize}
	\item A
	\item B
\end{itemize}
      %
\include{body/table}
%\pagenumbering{arabic}

\section{示例}

\subsection{插入图片}
如图\ref{fig:top_realmeaning}所示\\
\begin{figure}[thbp!]
\centering\includegraphics[width=0.9\linewidth]{figure/top_realmeaning.jpg}
\caption{ 模拟场景}
\label{fig:top_realmeaning}
\end{figure}

\subsection{插入代码}
\begin{lstlisting}[language=matlab]
  a=a;
  b=b;
  c=xxx;
\end{lstlisting}

\subsection{插入表格}
\begin{tabular}{cc}%一个c表示有一列,格式为居中显示(center)
(1,1)&(1,2)\\%第一行第一列和第二列  中间用&连接
(2,1)&(2,2)\\%第二行第一列和第二列  中间用&连接
\end{tabular}

\subsection{关于缩进}
\indent \textbf{xxxx}\\
\noindent xxxx

\subsection{list}
\begin{enumerate}
	\item A
	\item B
	\item C
\end{enumerate}

      %
%============= 参考文献 =====================
\addcontentsline{toc}{section}{参考文献}
\begin{thebibliography}{9}
\bibitem{latexcompanion} 
魏楚原 等.《大型园区网络建设与管理》.机械工业出版社.2015.3 
\bibitem{einstein} 
乔辉,刘晓辉,张新明 等.《网络硬件搭建与配置实践》.第三版.电子工业出版社.2012.5:第三章第二节
\bibitem{knuthwebsite} 



杜洪乐.《计算机网络实验指导书》.天津大学出版社.2016.3:
\end{thebibliography}
\clearpage

%=============  致谢  ======================
%\include{body/acknowledge}
%\newpage
\appendix

%%附录第一个章节
\section{心得体会}
\subsection{吕雪冰}
由于之前做过计网课设,所以对交换机的配置以及连接网线都熟悉了。这半天的收获是SUB接口配置,因为之前没有负责做这一部分。SUB子接口使一个物理接口上能多一个逻辑地址。我们的192.168.2.1和192.168.1.1子网通过192.168.3.1与192.168.3.2的子网互通,而这些只用了两个路由器上的各一个接口。节省了物理接口和网线。另外,由于我们组有三个同学做过课设,所以在设置虚拟子网和尝试用交换机连接主机时,完全就是照搬上学期做过的内容,但是组里的李昊同学改变子网掩码去连接,让我们在做完老师要求的之外能有机会进行别的尝试,这是我没能想到的,潜意识里觉得192.168是内部局域网地址,就该用这个了,是我的思维太固化了。

\subsection{张瀚林}
本次实验的主要内容是使用计算机、交换机等网络设备设置一个简单的局域网。通过课上老师对实验的内容讲解和观看实际的操作步骤以及同学们的帮助,学习到了一些新的知识,同时也对课本上的理论知识有了更加具体的认识。知道了在网络交换机级联方式不适当时,会出现广播风暴的问题以及出现这种问题的原因,具体表现是网络交换机相应的灯一直在闪烁。对于相关软件和软件的一些基本执行命令也有了一定程度的了解。同时,对局域网和虚拟局域网的相关设置方法和相关知识也有了一些初步的认识。

\subsection{凤雨婷}
	这次实验带给我的感觉和上课是的感觉不太一样, 上课时主要是考虑原理类的问题, 不怎么考虑细节, 而这次试验如果不考虑一些细节, 就做不出来, 当把这些细节搞明白后, 发现原来的理解还是有些不妥当的地方。 之前觉得这些细节不重要, 我只要知道原理, 具体实现可以让别人去做, 但现在想来, 只有这是两条腿走路, 缺一不可, 这些细节本质上是对整体架构的理解的投影, 如果有某些地方出了问题, 追更溯源一定是某个地方的理解还不够透彻。 实验的目的不是为了记住某些命令, 而是通过这样一种调试的方式, 来打通自己认知的漏洞。 而这种漏洞很难通过上课的方式来拟补的。 所以这次实验我的收获很大, 希望以后还能够经常参加这类实验。\\
	\indent 另外, 队友门的积极性也让我备受鼓舞, 因为大家都想把这个实验做好, 所以我自己的积极性也被调动了起来。


\subsection{李昊}
	我们组有三位同学之前都参加过硬件课设, 所以这次我们比较轻车熟路。
	这一次实验吸取上一次实验失败的经验, 我们在实验前就在纸上确定号我们需要的各种细节, 确定每个人都理解后, 分头行动, 合并的时候也很完美, 大家都知道要干什么。
	相比于之前的课设, 这次实验的反馈周期更短, 老师及时的纠正我们的错误, 我们没有深陷一些细节, 并且掌握了许多提高效率的技巧。
	比如
	\begin{enumerate}
	  \item tab可以补全, 而且命令只要前缀能够区分就不必完全写完
	  \item  通过执行 \emph{dispaly current-configeration} 命令可以方便的查看当前配置
	  \item 为了排除之前配置的影响, 可以通过\emph{reset}命令来清楚配置
	\end{enumerate}
	同时, 对计算机网络有了形而上学的一些感性认识, 比如
	\begin{enumerate}
	  \item \emph{上传下达, 单线联系, 各司其职}的原则
	  \item  原来路由器不用串口, 通过sub ip 也能够相连。 我们之前计网课设花了很长时间在纠结这个问题, 应为串口线接触不是很好
	\end{enumerate}
	最后, 感谢每一位同组的同学, 大家都想把这个实验做好, 所以我们才能够心往一处使, 高效的完成设计,实验,验证的一连串操作。


\subsection{王浩波}
这次的现代通信实验是搭建一个简单的网络模型,由于我在之前做过计网课设,所以见到这些路由器交换机并不会感到很陌生,无从下手,反而有一些亲切感,当初我们虽然也只是搭建了一个小型的校园网络,但是想到当初的努力和付出,最后实现了自己想要的结果,还是很开心,非常荣幸可以再次操作这些仪器,让我重新温习了当初的操作,有着自己的经验做支撑,老师交代的任务也很轻松地就完成了,我们组大家分工合作,每2到3个人一台电脑,进行IP的设置,路由器和交换机的设置,以及之后主机之间能否PING通的验证,每位同学都有操作的机会。这让我深刻体会到了团队合作的重要性,如果自己一个人只是配置交换机和路由器就会耗费自己许多时间。同时感谢老师全程的耐心指导,让我们更加顺利的完成了实验!



%============= 工作日志  ===================
%\includepdf[pages=1-10]{body/log}

\end{document}
%%%%%%%%%% 结束 %%%%%%%%%%
