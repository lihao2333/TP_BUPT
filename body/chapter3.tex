\section{个人感想}
\addtocounter{section}{1}
	\subsection*{李昊}
	以前一直对网络这一块有种神秘感, 很羡慕那些黑客们能够在网络世界中来去自如.通过这次实验更加认识到网络世界没有我想象的那么简单, 从每一个协议都可以看得出先辈们的精巧设计.这次实验虽然并没有想象中做的那么好, 但也算入了一点门了.\\
\indent	之前参加过一些团队比赛, 这次和大家合作是我感觉大家最积极的一次,只可惜我这一次作为组长并没有把大家的正真实力带出来,我们绝对有实力做的比现在这个状态要好.\\
\indent	我们遇到最大的困难就是资料和设备版本不一致,搜索合适的资源占用了我们大部分的时间\\
\indent	和别人沟通一定要明确, 比较好的方式就是将所要表达的在纸上清晰的呈现出来.\\
\indent	计划的重要性不在于要设计出来然后按部就班的执行, 而是明确出来要做什么后大家每个人心里都有个大概的数, 出了意外大家也会很默契的往一个方向上努力.
\indent 掌握一个快速测试的方法很重要,我们原先每次调试就是直接输入指令, 但是这样非常低效, 经常需要做重复的工作.后来我们将代码先写在文件中, 然后将整体代码复制到交互窗口中, 就能够批量执行了.这样一来修改也方便.
	\subsection*{王浩波}
    通过本次计算机网络的课程设计,我不仅对自己所学的计算机通信与网络的知识进行了巩固、验证和实践,来让自己对知识的理解进一步的加深与提高。同时整个工程的设计流程也让我受益颇多。我懂得了再要完成一个工程前,需要先做出一个整体的计划,在针对自己的计划去查找相关资料,依照自己查找到的资料进行一定的修改和完善,计划不仅要有实际的意义,还要合理可实现。\\
\indent    本次课程设计我们选择的项目为校园网的组建与应用,在课程开始之前,我们先在一起针对我们的这个课题进行了集中的讨论,来确定我们的组网计划,我们要做一个简单的小型校园网络,主体部分为一个三层交换机,通过一个路由器可以连接外网,三层交换机下面接二层交换机代表各个区域,再通过VLAN给这些区域划分开来并赋予实际的意义。\\
\indent 由于之前我并没有这一方面的知识储备,所以从开始课设的第一天我就积极做老师给的那一些教程,就是为了让自己在做项目时手足无措,老师推荐的那篇笑傲江湖对于我们这些没有接触过组网的人来说是一个很好的介绍,它告诉了我们网络的发展情况,同时也告诉了我们许多操作的作用和应用意义,这对于我们做项目时结合自己的实际项目背景来进行合理合适的操作有很大的帮助。在做这个项目的第一天,我对这些路由器和交换机还是很感兴趣的,通过超级终端可以查看这些仪器的配置,继续进行之后的操作,可以进行各种配置的修改,来使机器按照自己想要的方式来进行工作。\\
\indent    在项目进行的过程中我们遇到了很多的难题,最初的问题是两台路由器下的终端不能互相PING通,查找了许多资料,尝试了许多方法,结果是因为防火墙没有关闭;接下来就是给主机动态分配IP地址的问题,我们尝试了用一个DHCP服务器来给终端分配地址,但是只能做到分配一个网段的地址,再要继续尝试的时候,发现三层交换机可以在端口上用DHCP功能来分配IP地址,这样用起来更方便,快捷;最后的问题就是连接外网,我们先用一个路由器设置一个IP模拟外网,通过NAT协议,成功的在终端PING到了所用路由器,但是当用一台电脑连接无线网充当外网时,连接就出了问题,最终我们修改了连接外网的电脑的本体IP和路由器的一些设置才得以解决。\\
\indent    总之,这次课程设计是我以前从来没有过的经历,这对于我以后的学习与生活有很大的帮助,为我提供了很多的经验与方法,很感谢学校可以给我这样的机会!
	\subsection*{武宇}
\indent 	在本次课设之前,我还只是个对这些方面基本上一窍不通的小白,而课设后自己都觉得自己已经蜕变为了一只菜鸟!\\
\indent 	一共不到两周的时间,从第一周前几天研习小例子开始,不断地查阅相关资料、动手操作,逐渐的对这些设备由完全的陌生到稍微的认识。然而真知还是得出于实践,经过小一周的练习,我们开始正式对我们的工程进行建设。\\
\indent 	实战和练习的差别是很大的,在真正开始搭建我们的工程时,困难就一个接一个的来了。有很多地方和我们查阅到的资料是匹配不上的,比如配置指令这一部分。配置指令这一块着实影响了很多的进度,查阅到的配置指令有很多一部分无法在我们的仪器上实现。实际上主要还是因为我们并没有理解这些指令都有着怎样的意义和用法,最后发现其实没有的指令都拥有其他的指令进行替代。\\
\indent 	在指令用法上遇到的困难实际上还有因为我们对于一些知识概念的欠缺。连接三层交换机和路由器时我们操作了好久也没有将这两个仪器连接起来,最后查阅了很多资料后发现只是因为我们在三层交换机上添加的静态路由有误,改过来后一下就通了。\\
\indent 	在本次实验中团队配合也是极其重要的部分,大家各自分工,遇到困难彼此交流经验进行解决,大家群策群力总是能够激发出一个人无法得到的想法。\\
\indent 	虽然这次硬件课设的结果并没有达到我们预期的理想,但着实我们努力了,投入了大部分的精力在里面。在接下来的软件课设中期待大家的再次合作!!\\
	\subsection*{翁哲威}
在这个项目的实践中,我主要负责的是硬件的连接和调试,如二,三层交换机,路由器的连接和其协议设置,以及项目计划书的书写。在这个项目中,我学会了如何使用超级终端来设置交换机和路由器的协议设置,DHCP、VLAN、NAT、ACL这些原本在书上抽象的概念在实践中变得鲜活,是我对其有了很深刻的理解,得知了理论联系实践的重要性。这个项目同时让我知道了团队的力量,大家各取所长,共同为了一个目标奋斗,大家原先可能都对组网只有基础的了解且并没有实际的经验,但大家一起查资料,讨论,最终实现了目标。我从其他的组员身上学到了很多,加强了自身的知识。
	\subsection*{田玉}
	本次课设名额有限,我很幸运能够和大家一起坐在实验室中,共同实践自己从课本上学到的知识。两周时间其实很短,却也让我收获颇多。
\indent 一开始进入实验室,见到的是陌生的机器以及十几个G的学习资料。当时就觉得这个课设不会是个很简单的设计,后来也的确证实了这是个艰巨的实验任务,好在在小组成员的通力合作下,大家克服了一个又一个的困难,最后完成了实验任务。\\
\indent 为了让我们尽快熟悉设备的使用,老师给我们准备了几个实验,让我们通过做实验来熟悉华为Quidway系列的交换机和路由器的操作。尽管有了实验指导和工程指令指导,我们还是花了一星期的时间来熟悉设备。其中有一整天我们都在排查实验三中出现的问题。我们虽然按照了实验指导的步骤去设置了静态路由表,但是本应该互通的两个VLAN却一直ping不同,经过一步步排查,发现是电脑防火墙没有关闭。虽然在排查问题时花费了很多时间,但是我觉得正是这些时间里的探索,加深了我对知识点的了解。\\
\indent 在第一个星期的实践中,我最大的收获就是反复使用“?”,以及在之后跳出老师给的实验指导,去真真搭建自己校园网的过程中,我们使用最多的也是“?”。由于实验设备比较古老的缘故,在网上找到的教程中,很多命令都是找不到的,但我们通过“?”最后成功地设置了设备,完成我们校园网的搭建。\\
\indent 在此期间,十分感谢组员的细心教导与包容,感谢学校给我这次实践的机会,也感谢老师在实验期间的答疑。\\
	\subsection*{李佳澎}
\indent	这次为期十天的计网课设,让我真正意义上在现实中接触、体会、掌握了网络的相关知识。将脑海中抽象的路由器,交换机等设备概念,在现实中有了形象的认识,并进行了具体操作。对于IP的相关概念,例如IP类型、内、外网IP的区别、网络号、网段、网关、DHCP等、进行了实际的应用。\\
\indent	在计划书的策划阶段,包括校园网功能的设定,拓扑图的绘制等方面,大家集思广益,尽可能使之符合现实需要。最终,我们确定了路由器——三层交换机——二层交换机——主机的三层网络结构。并且将服务器连接置三层交换机,为主机提供网络服务。\\
\indent	到了设计的具体实施阶段,我才认识到理论知识、抽象设计和真正的项目、工程是有着天壤之别的。\\
\indent 项目与工程中的很多计划实施,需要对理论知识有一个很深入的了解,并且清楚其最适合的应用场景。比如在vlan划分的阶段,我们发现二层交换机所划分的vlan不能互通,而三层交换机vlan不能互通。后来查阅资料了解到,三层交换机能够自动实现vlan之间的连接,因此我们选择了在三层交换机上实现vlan,这样可以省去在二层交换机上划分vlan后用路由器桥接的麻烦。但是这样的vlan并不能实现访问隔离的功能。然后我们在老师的指导下,查阅相关资料,了解了很多实现这个功能的途径,比如更改端口类型、更改路由表、通过ACL做端口访问限制。\\
\indent	总之、这次课程设计令我受益匪浅,它让我明白理论知识固然重要,但是工程和项目所要达到的目的是符合实际需求。我们掌握的知识有限,很多时候不足以解决眼前的问题,所以需要我们发挥自己的主观能动性,在实践中学习理论知识,这也正是我们需要具备的重要素养。不要被自己已经掌握的知识限制了想象力,而要不断的去探索更好的解决办法。
	\subsection*{吴承隆}
	此次计网课设是入大学以来第一次较大规模的团队合作项目,通过对校园网的模拟搭建,进一步加深了我对路由器,交换机的概念理解,同时让我明白了理论学习与实践的缺一不可,实践中能迅速熟悉牢固理论知识。人类世界中网络的构建是极其复杂的,从本次我们小规模校园网的搭建中能深刻体会到这一道理,IP类型、内外网IP的、网络号、网关、DHCP、VLAN,人类为了方便快捷的生活绞尽脑汁,创造了一系列巧妙的解决方案,我深刻体会到我所掌握的知识仅是冰山一角,而为了进一步造福我们的生活,我们仍需专注于当下,巩固理论知识,通过实践去强化练习。同时,本次组网项目里,我体会到合作的重要性,组内成员分工明确,有条不紊,有利于项目的平稳进展,遇到难以攻克的难题时,要善于从网上、书中去寻找需要的帮助,要多查多问,不懂就要寻求问题的答案,成员之间要相互信任,我们组内的氛围就非常和睦,大家互帮互助,共同解决问题,在愉悦的氛围里士气也会高涨,有事半功倍之效。总之,感谢本次课设,让我对组网有了很大的收获和心得,对网络构建有了一定的意识框架,希望以后能有更多的机会接触相关项目。
	\subsection*{曹彪}
\indent 为期两周的《计算机网络课程设计》结束了,虽然时间不太长,但通过完成这两周的课程设计,我也收获到了很多东西,就像吴老师在第一天所说的,刚来实验室时的自己和两周后的自己有了很大的不同。这些东西都是其他的课程所收获不到的,总结起来有以下几点。\\
\indent 1.第一次用长达2周的时间做一件工程。课程设计过程中,老师也不断强调,要用看待工程的方式看待自己所做的东西,而不能以实验的方式。这也是我本次最大的收获。以前的很多实验课,基本都是用一下午时间,照着书本的内容进行操作,虽然有助于锻炼自己的动手实践能力,但从实际应用上来讲,意义很小;最主要的,这种实验,没法融入自己的想法,基本都按既定的套路操作,也没法融入实际的应用场景。而作为一名大三的学生,很可能在几年以后就要进入公司工作,到那时就不再有“实验”的机会,没人会告诉我们具体要怎么做,每个项目也都有所不同,我们所知道的只是项目需求和应用场景,这时候我们应当以工程角度看待问题,按照需求、应用场景、设备、资金等方方面面进行考量,融入自己的想法,给出合适的方案并加以实施。虽然课程设计最后做出来的成果不太满意,但我觉得我已经具备了一些参加实际工程的能力。\\
\indent 2.团队协作能力得到提升。我所在的小组有8个人,每个人的想法擅长的东西都不一样,每个人所做的工作量也不相同,如何配合协作好,高质量的完成任务就非常关键。在课程设计过程中,我们组内也有过几次意见不一致的地方,但最后都通过交流,找到了最佳的方案。每个人的所做的东西也都有些区别,至于我,主要在前期做了一些基础技术的铺垫准备工作,后期由于个人事情去的相对少一些,但都力求加入到项目制作过程中,给出一些自己的建议。通过这次课程设计我觉得,成功的、高效的团队应该有几个要点:一是最好要有一个团队的领导者,很好的将团队组织起来,有一定的决断能力去决定下一步该干什么,以后该干什么,这也是我一直欠缺的东西;二是要多交流,过程中由于个人理解想法不同,难免遇到意见不统一的地方,这时候就要多交流,将自己的想法告诉别人,交换意见,共同给出合适的决绝办法。三是每个人应当主动地加入进去,多做工作,有承担责任的意识;四是要有一定的技术储备,在此次课程设计之前,我们的技术储备是不够的。\\
\indent 3.学到很多实用的东西,对理论上的知识也有了一定的了解。在课程设计之前我也尝试看了看《计算机网络》课本,但是觉得内容十分抽象,晦涩难懂,比如说路由器,在课本上我知道路由器有ip地址,有路由表,有ospf和rip协议,有分组转发功能,但实际路由器接口什么样,怎么分配ip合适,路由表什么样,怎么增删路由表,怎么连接路由器和其他设备,这些都是在工程中会遇到的问题。同时我也学到了很多自己不太了解的知识,如vlan,nat,dhcp,dns,防火墙等知识,回过头来再看课本感觉很多东西都容易理解起来。还有,组内的其他成员做得一些工作我也都有了一定认识,例如服务器的架设,访问外网的技术等。但是我认为还是应该在进实验室之前有充足的理论基础的,这就不会再过程中耽误太多的时间。\\
\indent 4.对整体计算机网络的体系架构有了一定认识,尤其是对于分层的思想有了一些认识,我们平时看到的计算机网络,大多是应用层的东西,这次课程设计我认识计算机网络的下层也起到了十分关键的作用。例如二层的交换机,它所做的工作就只在二层,不涉及三层的ip部分,而三层的设备(如路由器、三层交换机)有路由的功能,想实现二层的互通或不互通可以在三层想办法。正是由于下面几层不可或缺的作用才给了应用层无限创新的可能性。\\
\indent 5.锻炼了我的意志品质。过程也遇到了一些障碍,想起来可行的东西,实现起来就是有问题,这时候就需要我们有良好的意志品质,马上冷静下来,理清思绪,寻找解决办法。\\
\indent 6.从一些失败的教训中取得了一些经验,以后做工程中应该尽量少走弯路,尽快确定自己想做什么,这样行动起来就不会盲目,还有想问题要想的细致一些,有时候自己想的不够细会导致浪费很多时间。\\
