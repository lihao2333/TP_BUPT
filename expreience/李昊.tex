\subsection{李昊}
	我们组有三位同学之前都参加过硬件课设, 所以这次我们比较轻车熟路。
	这一次实验吸取上一次实验失败的经验, 我们在实验前就在纸上确定号我们需要的各种细节, 确定每个人都理解后, 分头行动, 合并的时候也很完美, 大家都知道要干什么。
	相比于之前的课设, 这次实验的反馈周期更短, 老师及时的纠正我们的错误, 我们没有深陷一些细节, 并且掌握了许多提高效率的技巧。
	比如
	\begin{enumerate}
	  \item tab可以补全, 而且命令只要前缀能够区分就不必完全写完
	  \item  通过执行 \emph{dispaly current-configeration} 命令可以方便的查看当前配置
	  \item 为了排除之前配置的影响, 可以通过\emph{reset}命令来清楚配置
	\end{enumerate}
	同时, 对计算机网络有了形而上学的一些感性认识, 比如
	\begin{enumerate}
	  \item \emph{上传下达, 单线联系, 各司其职}的原则
	  \item  原来路由器不用串口, 通过sub ip 也能够相连。 我们之前计网课设花了很长时间在纠结这个问题, 应为串口线接触不是很好
	\end{enumerate}
	最后, 感谢每一位同组的同学, 大家都想把这个实验做好, 所以我们才能够心往一处使, 高效的完成设计,实验,验证的一连串操作。

